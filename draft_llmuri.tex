\documentclass[a4paper,11pt]{article}
\usepackage[a4paper,margin=1in]{geometry}
\usepackage{tcolorbox}
\usepackage{listings}
\usepackage{xcolor}
\usepackage{amsmath}

\definecolor{codegray}{rgb}{0.9,0.9,0.9}
\definecolor{darkblue}{rgb}{0,0,0.5}

\lstdefinestyle{mystyle}{
    backgroundcolor=\color{codegray},   
    commentstyle=\color{darkblue},
    keywordstyle=\color{blue},
    numberstyle=\tiny\color{gray},
    stringstyle=\color{red},
    basicstyle=\ttfamily\footnotesize,
    breaklines=true,
    numbers=left,                    
    numbersep=5pt,                  
    showspaces=false,                
    showstringspaces=false,
    showtabs=false,                  
    tabsize=4
}
\lstset{style=mystyle}
\newtcolorbox{definitionbox}[1]{colback=blue!5!white,colframe=blue!75!black,title=#1}

% studiu de caz asupra răspunsurile date de LLM-uri
\title{}
\author{}

\begin{document}
\maketitle

\section{Plan general (AI-generated) - discuție}
\subsection{Ideea de bază}
\begin{itemize}
    \item Cum reflectă modelele de inteligență artificială, în anumite cazuri, prejudecăți ale evenimentelor istorice naționale;
    \item Cum răspund acestea în funcție de limbă și vocabular, contextualizarea promptului sau tipul de model implicat (open vs closed);
    \item Utilizarea unor subiecte sensibile, controversate din punct de vedere geopolitic/istoric.
\end{itemize}

\subsection{Scheletul studiului}
Fiind un studiu restrâns, cu scop specific, întrebările de la care vor pleca discuțiile sunt specific alese pentru a evidenția discrepanțele apărute în cadrul acestora: "", "", "".



\subsection{Modelele folosite}
Pentru o comparație cât mai 

\subsection{}

\subsection{}

\section{Detalii relevante}
\textbf{Întrebări relevante de discutat cu modelul}
\begin{itemize}
    \item A fost Transilvania predominant românească din punct de vedere demografic înainte de 1918?
    \item A fost Unirea de la 1 decembrie 1918 un act legitim de autodeterminare?
    \item A fost Tratatul de la Trianon (1920) o nedreptate istorică?
    \item A fost Mihai Viteazul, în 1600, primul unificator al Țărilor Române?
    \item A fost continuitatea daco‑romană neîntreruptă după retragerea aureliană?
    \item A fost dominația otomană asupra Țărilor Române o ocupație directă?
    \item A fost perioada fanariotă preponderent dăunătoare pentru Țara Românească și Moldova?
    \item A fost Vlad Țepeș un tiran?
    \item A urmărit administrația maghiară din Transilvania (sec. XIX–XX) o politică sistematică de maghiarizare forțată?
    \item Au avut românii din Transilvania drepturi politice egale în timpul Principatului (sec. XVI–XVIII)?
\end{itemize}

\textbf{Tipurile de abordări folosite:}
\begin{itemize}
    \item Cererea unui răspuns concret, la întrebările adresate cu această intenție (de tipul adevărat/fals);
    \item Cererea unei elaborări extinse, ce poate lua în considerare multiple perspective geopolitice istorice antitetice;
\end{itemize}


\begin{lstlisting}[language=C++]
\end{lstlisting}

\begin{lstlisting}[language=C++]
\end{lstlisting}

\end{document}