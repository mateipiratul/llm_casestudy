\documentclass[a4paper, 10pt, twocolumn]{article}
\usepackage[romanian]{babel}
\usepackage[utf8]{inputenc}
\usepackage[T1]{fontenc}
\usepackage[a4paper, top=2cm, bottom=2cm, left=1.5cm, right=1.5cm, columnsep=1cm]{geometry}
\usepackage{graphicx}
\usepackage{hyperref}
\usepackage{authblk}

\hypersetup{
    colorlinks=true,
    linkcolor=blue,
    urlcolor=blue,
    pdfauthor={Numele Vostre},
    pdftitle={Titlul Studiului Vostru}
}
\title{\Huge Analiza Comparativă a Biasului și Inconsecvenței în Răspunsurile Modelelor Lingvistice Mari (LLM)}
\author[1]{}
\author[2]{}
\author[2]{Cocu Matei-Iulian}

\affil[1]{Facultatea de Matematică și Informatică, Universitatea din București}
\affil[ ]{\textit{\{,\}@s.unibuc.ro, }}
\date{\today}
\begin{document}
\maketitle

\begin{abstract}
Utilizarea modelelor lingvistice mari (LLM-uri) în domeniile umanistice a devenit o obișnuință, dată fiind evoluția și ușurința de utilizare a acestora, unul dintre aceste domenii fiind rescris și reinterpretat, în particular, în funcție de interesele și motivele celor implicați - istoria. În acest studiu de caz, multiple modele lingvistice mari, atât open-source cât și closed-source, au oferit diverse răspunsuri anumitor întrebări de natură controversată ce țin de istoria României pe plan extern, din diferite perioade seculare. Motivația constă în recunoașterea faptului că istoria este adesea prezentată prin prisma unor perspective alterate, influențate primordial de cultura și idealurile unui stat, chiar și prin intermediul modelelor lingvistice mari. Fiind de cele mai multe ori antrenate pe anumite seturi de date ce pot prezenta anumite ambiguități, lipsa de neutralitate este insuflată ulterior utilizatorilor, contribuind astfel în mod indirect la perpetuarea informației incorecte. Procesul de cercetare a fost desfășurat în trei etape, cu scopul de a confirma ideea că tipul de răspuns așteptat poate influența, până într-un punct, răspunsul oferit în sine. Cercetările noastre aduc la suprafață predispunerea modelelor la asemenea inconsistențe, în cadrul unei contextualizări specifice a limbajului pentru întrebarea pusă. Codul sursă poate fi găsit pe \href{https://github.com/mateipiratul/llm_studycase}{GitHub}.
\end{abstract}

\section{Introducere}
Un număr semnificativ de studii indică faptul că pe parcursul ultimei perioade de timp modelele (reasoning models and so on).
[Istoria - subiect controversat; referinta paper - gandirea critica a llmurilor este similara cu cea a vesticilor; 
\ref{fig:workflow}.

\begin{figure}[h!]
    \centering
    % \includegraphics[width=\columnwidth]{}
    \caption{}
    \label{fig:workflow}
\end{figure}

\section{Metodologia Studiului}

\subsection{Colectarea Datelor}

\subsection{Criterii de Analiză}

% utila pentru diagrame mari
\begin{figure*}[t!]
    \centering
    % \includegraphics[width=\textwidth]{}
    \caption{Exemplu comparativ de răspunsuri de la cele cinci modele la întrebarea despre Revoluția Industrială. Se observă diferențe notabile în accentul pus pe consecințele sociale versus cele tehnologice.}
    \label{fig:wide_comparison}
\end{figure*}

\section{Analiza Răspunsurilor și Rezultate}

\section{Discuții despre Bias și Inconsecvențe}

\section{Concluzii}

\section*{Referințe}
\begin{thebibliography}{9}
    \bibitem{}
    text
    \textit{}.

    \bibitem{}
    text
    \textit{}.
\end{thebibliography}

\end{document}