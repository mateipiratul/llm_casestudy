\documentclass[a4paper, 10pt, twocolumn]{article}
\usepackage[romanian]{babel}
\usepackage[utf8]{inputenc}
\usepackage[T1]{fontenc}
\usepackage[a4paper, top=2cm, bottom=2cm, left=1.5cm, right=1.5cm, columnsep=1cm]{geometry}
\usepackage{graphicx}
\usepackage{hyperref}
\usepackage{authblk}

\hypersetup{
    colorlinks=true,
    linkcolor=blue,
    urlcolor=blue,
    pdfauthor={Numele Vostre},
    pdftitle={Titlul Studiului Vostru}
}
\title{\Huge Analiza Comparativă a Biasului și Inconsecvenței în Răspunsurile Modelelor Lingvistice Mari (LLM)}
\author[1]{}
\author[2]{}
\author[1]{}

\affil[1]{Facultatea de Matematică și Informatică, Universitatea din București}
\affil[ ]{\textit{\{,\}@s.unibuc.ro, }}
\date{\today}
\begin{document}
\maketitle

\begin{abstract}

\end{abstract}

\section{Introducere}
\ref{fig:workflow}.

\begin{figure}[h!]
    \centering
    \includegraphics[width=\columnwidth]{placeholder.png}
    \caption{}
    \label{fig:workflow}
\end{figure}

\section{Metodologia Studiului}

\subsection{Colectarea Datelor}

\subsection{Criterii de Analiză}

% utila pentru diagrame mari
\begin{figure*}[t!]
    \centering
    \includegraphics[width=\textwidth]{placeholder-image-wide.png}
    \caption{Exemplu comparativ de răspunsuri de la cele cinci modele la întrebarea despre Revoluția Industrială. Se observă diferențe notabile în accentul pus pe consecințele sociale versus cele tehnologice.}
    \label{fig:wide_comparison}
\end{figure*}

\section{Analiza Răspunsurilor și Rezultate}

\section{Discuții despre Bias și Inconsecvențe}

\section{Concluzii}

\section*{Referințe}
\begin{thebibliography}{9}
    \bibitem{}
    text
    \textit{}.

    \bibitem{}
    text
    \textit{}.
\end{thebibliography}

\end{document}